\documentclass[11pt,a4paper]{article}
\usepackage[utf8]{inputenc}
\usepackage{amsmath}
\usepackage{amsfonts}
\usepackage{amssymb}
\usepackage[left=2cm,right=2cm,top=2cm,bottom=2cm]{geometry}


\usepackage{slashed}
\usepackage{hyperref}
\usepackage{graphicx}
\usepackage{caption}
\usepackage{float}

\usepackage{subcaption}
\usepackage{cancel}


\hypersetup{
	%	bookmarks=true,         % show bookmarks bar?
	unicode=true,          % non-Latin characters in Acrobat's bookmarks
	pdftoolbar=true,        % show Acrobat's toolbar?
	pdfmenubar=true,        % show Acrobat's menu?
	pdffitwindow=false,     % window fit to page when opened
	pdfstartview={FitH},    % fits the width of the page to the window
	pdftitle={NSCFI},    % title
	pdfauthor={Author},     % author
	pdfsubject={Subject},   % subject of the document
	pdfcreator={},   % creator of the document
	pdfproducer={Producer}, % producer of the document
	pdfkeywords={dark matter} {freeze-in} {non standard cosmology}, % list of keywords
	pdfnewwindow=true,      % links in new PDF window
	colorlinks=true,       % false: boxed links; true: colored links
	linkcolor=blue,          % color of internal links (change box color with linkbordercolor)
	citecolor=red,        % color of links to bibliography
	filecolor=black,      % color of file links
	urlcolor=blue           % color of external links
}

%There are some definitions of macros in "macros.tex". So, import them!
\usepackage{ifthen}
\usepackage{tikz}

%Define blue color environment:
\usepackage{color}
\newenvironment{bl}[1]{{\color{blue} #1}}



\newcommand{\tM}[2]{ |\tilde{ \mathcal{M} }_{#1 \to #2}|^2}

\newcommand{\vev}[1]{\langle #1 \rangle}
\newcommand{\Bvev}[1]{\Bigg\langle #1 \Bigg\rangle}
\newcommand{\bvev}[1]{\Big\langle #1 \Big\rangle}

\newcommand{\bra}[1]{\langle #1 |}
\newcommand{\ket}[1]{| #1 \rangle}




\newcommand{\dint}{  \displaystyle \int }
%%%%%%%%%%%%%%%%%%%%%%%%%%%%%%%%%%%%%%%%%%
\newcommand{\ie}{{\em i.e.} }
\newcommand{\eg}{{\em e.g.} }
\newcommand{\GeV}{{\rm GeV}}
\newcommand{\TeV}{{\rm TeV}}
\newcommand{\MeV}{{\rm MeV}}
\newcommand{\keV}{{\rm keV}}
\newcommand{\rhs}{RHS }
\newcommand{\lhs}{LHS }



\newcommand{\geff}{ g_{\rm eff} }
\newcommand{\heff}{ h_{\rm eff} }


\newcommand{\Lint}{ \mathcal{L}_{\rm int} }
 


\newcommand{\lrb}[1]{\left( #1 \right)}
\newcommand{\lrsb}[1]{\left[ #1 \right]}
\newcommand{\lrBigb}[1]{\Big( #1 \Big)}
\newcommand{\lrBigsb}[1]{\Big[ #1 \Big]}
\newcommand{\lrBiggb}[1]{\Bigg( #1 \Bigg)}
\newcommand{\lrBiggsb}[1]{\Bigg[ #1 \Bigg]}

\newcommand{\lrBigcb}[1]{\Big\{ #1 \Big\}}
\newcommand{\lrBiggcb}[1]{\Bigg\{ #1 \Bigg\}}
%%%%%%%%%%%%%%%%%%%%%%%%%%%%%%%%%%%%%%%%%

%%%%%%%%%%%%%%%%%%%%%%%%%%%%%%%%%%%%%%%%%%%%%%%%%%%--Begin_refs--%%%%%%%%%%%%%%%%%%%%%%%%%%%%%%%%%%%%%%%%%%%%%%%%%%%%%%%%%%%%%%%%%%%%%%
\newcounter{NumArgs}

%Define reference to an arbitrary number of equations (\eqs{label_1,label_2....,label_n} will show eqs. ref_1, ref_2, ..., and ref_n)
\newcommand{\eqs}[1]{\setcounter{NumArgs}{0}\foreach\i in{#1}{\stepcounter{NumArgs}}%
	\ifthenelse{\equal{\theNumArgs}{1}}{eq.~(\ref{#1})}%
	{\ifthenelse{\equal{\theNumArgs}{2}}%
		{eqs.~\foreach\i[count=\q]in{#1}{\ifthenelse{\equal{\q}{\theNumArgs}}{and (\ref{\i})}{(\ref{\i})~}}}%
		{eqs.~\foreach\i[count=\q]in{#1}{\ifthenelse{\equal{\q}{\theNumArgs}}{and (\ref{\i})}{(\ref{\i}),~}}}}}


%Define reference to an arbitrary number of equations (\Eqs{label_1,label_2....,label_n} will show Eqs. ref_1, ref_2, ..., and ref_n)
\newcommand{\Eqs}[1]{\setcounter{NumArgs}{0}\foreach\i in{#1}{\stepcounter{NumArgs}}%
	\ifthenelse{\equal{\theNumArgs}{1}}{Eq.~(\ref{#1})}%
	{\ifthenelse{\equal{\theNumArgs}{2}}%
		{Eqs.~\foreach\i[count=\q]in{#1}{\ifthenelse{\equal{\q}{\theNumArgs}}{and (\ref{\i})}{(\ref{\i})~}}}%
		{Eqs.~\foreach\i[count=\q]in{#1}{\ifthenelse{\equal{\q}{\theNumArgs}}{and (\ref{\i})}{(\ref{\i}),~}}}}}


%Define reference to an arbitrary number of labels (\REF{label_1,label_2....,label_n} will show ref_1, ref_2, ..., and ref_n)
\newcommand{\refs}[1]{\setcounter{NumArgs}{0}\foreach\i in{#1}{\stepcounter{NumArgs}}%
	\ifthenelse{\equal{\theNumArgs}{1}}{(\ref{#1})}%
	{\ifthenelse{\equal{\theNumArgs}{2}}%
		{\foreach\i[count=\q]in{#1}{\ifthenelse{\equal{\q}{\theNumArgs}}{and (\ref{\i})}{(\ref{\i})~}}}%
		{\foreach\i[count=\q]in{#1}{\ifthenelse{\equal{\q}{\theNumArgs}}{and (\ref{\i})}{(\ref{\i}),~}}}}}



%Define reference to an arbitrary number of figs (\Figs{label_1,label_2....,label_n} will show ref_1, ref_2, ..., and ref_n)
\newcommand{\Figs}[1]{\setcounter{NumArgs}{0}\foreach\i in{#1}{\stepcounter{NumArgs}}%
	\ifthenelse{\equal{\theNumArgs}{1}}{Fig.~(\ref{#1})}%
	{\ifthenelse{\equal{\theNumArgs}{2}}%
		{Figs.~\foreach\i[count=\q]in{#1}{\ifthenelse{\equal{\q}{\theNumArgs}}{and (\ref{\i})}{(\ref{\i})~}}}%
		{Figs.~\foreach\i[count=\q]in{#1}{\ifthenelse{\equal{\q}{\theNumArgs}}{and (\ref{\i})}{(\ref{\i}),~}}}}}




%Define reference to an arbitrary number of "general reference" (\Gen{message}{label_1,label_2....,label_n} will show message.(ref_1), (ref_2), ..., and (ref_n)
\newcommand{\Gen}[2]{\setcounter{NumArgs}{0}\foreach\i in{#2}{\stepcounter{NumArgs}}%
	\ifthenelse{\equal{\theNumArgs}{1}}{#1.~(\ref{#2})}%
	{\ifthenelse{\equal{\theNumArgs}{2}}%
		{#1.~\foreach\i[count=\q]in{#2}{\ifthenelse{\equal{\q}{\theNumArgs}}{and (\ref{\i})}{(\ref{\i})~}}}%
		{#1.~\foreach\i[count=\q]in{#2}{\ifthenelse{\equal{\q}{\theNumArgs}}{and (\ref{\i})}{(\ref{\i}),~}}}}}


%%%%%%%%%%%%%%%%%%%%%%%%%%%%%%%%%%%%%%%%%%%%%%%%%%%--End_refs--%%%%%%%%%%%%%%%%%%%%%%%%%%%%%%%%%%%%%%%%%%%%%%%%%%%%%%%%%%%%%%%%%%%%%%%so I have to make it a tex, and load this it order for it to pickup the commands...
%NOTE: in case you define new macros, put them, and the required packages, in there.

\usepackage{authblk}


\author[ ]{Dimitrios Karamitros}



\affil[ ]{  }
\affil[ ]{\textit{E-mail: }
	\href{mailto:dkaramit@yahoo.com}{\bl{dkaramit@yahoo.com}},
	\href{mailto:dimitrios.karamitros@ncbj.gov.pl}{\bl{dimitrios.karamitros@ncbj.gov.pl}},
}

\title{The Boltzmann Equation and ambiguous factors}

\renewcommand{\theequation}{\arabic{section}.\arabic{equation}}

%detele this to have section.equation
%\renewcommand{\theequation}{\arabic{equation}}

\begin{document}

\maketitle
\begin{abstract} 
We show the correct way to think of and use Boltzmann equations. 
\end{abstract}
\flushbottom

\section{Introduction}\label{sec:intro}
\setcounter{equation}{0}
%
In the literature, the Boltzmann equation (BE) is usually presented for specific simple processes such as $1 \leftrightarrow 2$ or  $2 \leftrightarrow 2$, and then assumed that the generalization is obvious. This is not the case at all, and one needs to specify the rules of the game, so that  BEs describing a system  are correctly implemented.

\subsection{What {\em is} a Boltzmann equation}
General form of a BE for the evolution of the phase-space distribution (PSD) of a particle $\phi$ is 
%
\begin{equation}
\dfrac{d}{dt} f_{\phi} (\vec{p}) = \sum_{A} \sum_{B} \ I_{A,B}(\vec{p}) \;, 
\label{eq:GeneralBE}
\end{equation}
%
with the sums running in {\em all} possible processes that involve $\phi$ with initial states $A$ and final states $B$. The collision integral $I_{A,B}$, basically, is the rate at which $\phi$ states are created and destroyed. 

To understand what the BE says, we start with the PSD, which is
%
\begin{equation}
	\dfrac{d^3 \vec{x} \ d^3 \vec{p}}{(2\pi)^3} \ f(\vec{p})  \equiv \# \text{particles (or states) inside volume $d^3 \vec{x}$ with momentum $d^3 \vec{p}$} \;.
	\label{eq:PSD_description}
\end{equation}
%
Then we can think of the BE as  
%
\begin{align}
	\dfrac{d^3 \vec{x} \ d^3 \vec{p}}{(2\pi)^3}	\dfrac{d}{dt}  \ f(\vec{p})  &= 
	\text{rate of creation of particles (or states) in  inside volume $d^3 \vec{x}$ with momentum $d^3 \vec{p}$} \nonumber \\
	&-\text{rate of destruction of particles (or states) in  inside volume $d^3 \vec{x}$ with momentum $d^3 \vec{p}$} \;.
	\label{eq:BE_description}
\end{align}
%
The transition rate per unit volume is 
%
$$
\dfrac{d}{dt dV} P(i\to f) \equiv \prod_{i=1}^{n} \dfrac{1}{2E_{i}} \ \prod_{i=1}^{m} \dfrac{d^3\vec{k}_f}{(2\pi)^3 \ 2E_{f}}  w(i\to f) 
f_{\rm fin}(t,\vec{k}_f)  \;,
$$
%
with $w(i\to f)$ the transition matrix, and $f_{\rm fin}(t,\vec{k}_f)$ a function that takes into account the occupation number. Before moving to the actual creation and destruction rates, we should first discuss $f_{\rm fin}$. This is different for bosons, fermion, and classical particles. For classical particles $f_{\rm fin} = 1$, since it doesn't matter whether a state is occupied or not. For fermions, we have $f_{\rm fin}(t,\vec{k}_f) =\lrb{1 -  f(t, \vec{k}_f ) }$, while for bosons $f_{\rm fin}(t,\vec{k}_f) = \lrb{1 +  f(t, \vec{k}_f) }$.


\subsubsection*{Bosonic states}
%
For the bosonic case, we show a simple example in order to discover what is called {\em bose enhancement}. Consider a system with two states ($\ket{1,2}$) and $N+1$ particles, and a transition operator
%
$$
\hat H = U \sum_{i=1}^{N+1} \ket{1}_i \bra{2}_i \;.
$$
%
Consider an initial state $\ket{\psi} = \dfrac{1}{\sqrt{N+1}} \lrBiggcb{ \ket{2}_1   \prod_{i=2}^{N+1} \ket{1}_i + 
	\ket{2}_2 \ket{1}_1 \prod_{i=3}^{N+1} \ket{1}_i  + \cdots +  \ket{2}_{N+1} \prod_{i=1}^{N} \ket{1}_i } $, \ie a state where
one particle is in $\ket{2}$ and all others in $\ket{1}$.

The probability of one a transition to $\ket{\chi} = \prod_{i=1}^{N+1} \ket{1}_i$ (\ie all states to $\ket{1}$) is
%
\begin{align*}
	|\bra{\chi} \hat{H}  \ket{\psi} |^2 = (N+1) |U|^2 \;.
\end{align*}
%
This means that the probability increases with the number of occupation!

\subsubsection*{Fermionic states}
%
Similarly, assuming two particles in a two-state system with a transition matrix 
%
$$
\hat H = U \lrb{ \ket{1}_1 \bra{2}_2 + \ket{2}_1 \bra{1}_2}\;,
$$
% 
and an initial state $ \ket{\psi} = \dfrac{1}{\sqrt{2}} \lrb{  \ket{1}_1 \ket{2}_2 - \ket{2}_1 \ket{1}_2  }  $, we find that
the probability to transition to the state $\ket{\chi} = \ket{1}_1\ket{1}_2 $ is
%
\begin{align*}
	|\bra{\chi} \hat{H}  \ket{\psi} |^2 = (1-1) |U^2| = 0\;.
\end{align*}
%
Notice that a better illustration would be to show what happens with $N$-state system of $M \leq N$ particles, but the result would be the same since
each state would have a maximum occupation number of $1$.



\subsection*{Creation and distraction rates}
%
Let's assume we keep track of a specific system inside a phase-space volume element,$d^3 \vec{x} \ d^3 \vec{p}$, in order to find how the occupation
number evolves by interaction with another system.  To do this, we need to average over all states, except the one we are looking for. Meaning that the 
creation rate is
%
\begin{align}
	\text{Creation rate of states inside $d^3 \vec{x} \ d^3 \vec{p}$ } =  
	 \dfrac{d^3 \vec{x} \ d^3 \vec{p}}{(2\pi)^3} \dfrac{1}{2E_p} 
	\dint  \prod_{i=1}^{n} \dfrac{d^3 \vec{p}_i}{(2\pi)^3 \ 2E_{i}} \ \prod_{i=1}^{m} \dfrac{d^3\vec{k}_f}{(2\pi)^3 \ 2E_{f}}  w_{\vec{p}}(i\to f) 
	\ f(t,\vec{p}, \vec{p}_i , \vec{k}_f)   \; ,
	\label{eq:creation_rate}
\end{align}
%
where we do not integrate over $d^3 \vec{p}$, since we keep track inside this momentum range. This is the reason $w_{\vec{p}}( i\to f)$ depends on $\vec{p}$
explicitly.  Keep in mine that $f(t,\vec{p}, \vec{p}_i , \vec{k}_i)$ is used in order to average the number on transition taking place according to the 
occupation number of all states.

In a similar fashion, we find the distraction rate to be the same by changing  $i \leftrightarrow f$, \ie 
%
\begin{align}
	\text{Distraction rate of states inside $d^3 \vec{x} \ d^3 \vec{p}$ } =  
	\dfrac{d^3 \vec{x} \ d^3 \vec{p}}{(2\pi)^3} \dfrac{1}{2E_p} 
	\dint  \prod_{i=1}^{n} \dfrac{d^3 \vec{p}_i}{(2\pi)^3 \ 2E_{i}} \ \prod_{i=1}^{m} \dfrac{d^3\vec{k}_f}{(2\pi)^3 \ 2E_{f}}  w_{\vec{p}}(f \to i) 
	\ \tilde{f}(t,\vec{p}, \vec{p}_i , \vec{k}_f)   \; ,
	\label{eq:distraction_rate}
\end{align}
%
where we note that in principle $f$ and $\tilde{f}$ can be different. 

Both $f$ and $\tilde{f}$ are
%
\begin{align*}
	&f(t,\vec{p}, \vec{p}_i , \vec{k}_f) = f_{\rm ini}(t,\vec{p}, \vec{p}_i) \times f_{\rm fin}(t, \vec{k}_f)  \\
	&\tilde f(t,\vec{p}, \vec{p}_i , \vec{k}_f) = \tilde f_{\rm ini}(t, \vec{k}_f) \times \tilde f_{\rm fin}(t,\vec{p}, \vec{p}_f) \;,
\end{align*}
%
We note that the contribution due to the initial states can be decomposed into the product of the SPDs of the initial states, \ie
%
\begin{align*}
f_{\rm ini}(t, \vec{p},\vec{p}_i) &= f(t, \vec{p}) \times \prod_{i} f(t, \vec{p}_i)  \\
\tilde f_{\rm ini}(t, \vec{k}_f) &= \prod_{f} f(t, \vec{k}_f) \;.
\end{align*}


The final state SPD decomposition is the product of the different contributions depending on whether the statistics of the particles.That is
%
\begin{align*}
	f_{\rm fin}(t, \vec{p},\vec{p}_i) &= \lrb{ 1 +  \epsilon \   f(t, \vec{p}) } \times \prod_{i} \lrb{ 1 +  \epsilon \  f(t, \vec{p}_i) } \\
	\tilde f_{\rm fin}(t, \vec{k}_f) &= \prod_{f} \lrb{ 1 +  \epsilon \ f(t, \vec{k}_f) } \;,
\end{align*}
%
with $\epsilon = +1$ for bosons, $\epsilon = -1$ for fermions, and $\epsilon = 0$ for classical particles. 


\subsubsection*{Definitions}

Before moving on, let's define some useful things.  In order to make things easier, we introduce the ``enhanced" matrix element squared for a 
process $A \to B$ 
%
\begin{equation}
	w(A \to B) = |\tilde{ \mathcal{M} }_{A \to B}|^2 = (2\pi)^4 \ \delta^{(4)}\lrb{\sum_A p_A - \sum_B p_B}  \  \sum_{\rm idofs}|\mathcal{M}_{A \to B}|^2 \; ,
	\label{eq:tildeM}
\end{equation}
%
and the initial and final phase space volume element as 
%
\begin{eqnarray}
	&d \Pi_i  \equiv \dfrac{d^3 \vec{q}}{ (2 \pi)^3 } \dfrac{1}{2\omega_q} \ f_{i}(\vec{q}) \nonumber \\ 
	&d \Phi_i  \equiv \dfrac{d^3 \vec{q} }{ (2 \pi)^3 } \dfrac{1}{2\omega_q} \ \lrb{1 \pm f_{i}(\vec{q})}
	\label{eq:PhaseSpaceVolume}
\end{eqnarray}
% 

\subsection{The final form}
%
The BE, finally, takes the form
%
\begin{align*}
	\dfrac{d^3 \vec{x} \ d^3 \vec{p}}{(2\pi)^3}	\dfrac{d}{dt}  \ f(\vec{p})=&
		\lrb{  1+ \epsilon f(t,\vec{p} )} \dfrac{d^3 \vec{x} \ d^3 \vec{p}}{(2\pi)^3} \dfrac{1}{2E_p} 
	\dint  \prod_{i=1}^{n} d \Pi_i \ \prod_{i=1}^{m} d \Phi_f 
	\ w_{\vec{p}}(i \to f)    \\
	%	
	-&f(t,\vec{p} ) \dfrac{d^3 \vec{x} \ d^3 \vec{p}}{(2\pi)^3} \dfrac{1}{2E_p} 
	\dint  \prod_{i=1}^{n} d \Phi_i \ \prod_{i=1}^{m} d \Pi_f 
	\ w_{\vec{p}}(f \to i)  \; ,
\end{align*}
%
or
%
\begin{align}
	2E_p	\dfrac{d}{dt}  \ f(\vec{p})=&	 
	\lrb{  1+ \epsilon f(t,\vec{p} )} \ \dint  \prod_{i=1}^{n} d \Pi_i \ \prod_{i=1}^{m} d \Phi_f 
	\ w_{\vec{p}}(i \to f)    \nonumber \\
	%	
	-&
	f(t,\vec{p} ) \ \dint  \prod_{i=1}^{n} d \Phi_i \ \prod_{i=1}^{m} d \Pi_f 
	\ w_{\vec{p}}(f \to i)  \; .
	\label{eq:BE_general}
\end{align}



\section{How to figure out the correct factors}\label{sec:Factors}
\setcounter{equation}{0}
%
In \eqs{eq:BE_general}, we do not mention whether $i$ or $f$ belong to the species of the f we track down. Here, we try to address that.
\subsection*{A simple $2 \to 2$ process}
Suppose that in the system under study there exist only two particles, $\chi$ and $\psi$, with the only interactions being $\chi \psi \leftrightarrow \chi \chi$. The evolution of the phase-space distribution of $\psi$ is given with no ambiguity on what factors we should use, because there is only one $\psi$ created and destroyed in such processes. Labeling the momenta as  $$\chi(q_\chi) \psi(p) \leftrightarrow \chi(q_1)\chi(q_2) \;,$$ the BE becomes
%
\begin{equation}
\dfrac{d}{dt} f_{\psi }(\vec{p}) = \dfrac{1}{2E_p} \lrsb {
\dint  \lrb{\dfrac{1}{2} d\Pi_{1} d\Pi_{2}} d\Phi_{\chi}  \tM{\chi \chi }{\chi \psi} \lrb{1\pm f_{\psi }(\vec{p})}
-\dint d\Pi_{\chi} \lrb{\dfrac{1}{2} d\Phi_{1} d\Phi_{2}}   \tM{\chi \psi}{\chi \chi } f_{\psi }(\vec{p} )
}\;.
\label{eq:BE_22-psi}
\end{equation}
%
Notice that there is a factor of $\dfrac{1}{2}$ since we integrate  over the momenta of two identical particles. This is a simple and quite natural BE, with the only factors being the symmetry factors that should be there. 

The problem arises when we try to write down a BE for $f_\chi$. The problem seems to be on the factors should we put. Surely, we don't create one particle but two... or maybe one? How do we count the ``$\chi$ states" we create in each process? A simple solution would be to say that we put the usual symmetry factors, but multiply with the net number of particles created. This can't be true, though. Consider for example a process like $\chi \chi \to \chi \chi$, which re-distributes energy in the system. Are all elastic scattering contributions vanish identically? This would mean that an isolated system of strongly integrating identical particles never reaches thermodynamic equilibrium. I think that Boltzmann would disagree with this. 

So we need to think of what the BE says. The BE keeps track of $\chi$ states. Not one particular particle $\chi$, but all of them participating in all possible interactions. But it keeps track of the states created, one at a time. That is, in one process we only see what one $\chi$ that participates in the process does. If the $\chi$ we keep track  is created, then we add a $\chi$ state the system, otherwise we take away one. So, in the system at hand, both $\chi \psi \to \chi \chi $ and $\chi \chi \to \chi \psi $ can create and destroy $\chi$ states. Before writing the BE, we should choose how to label the momenta. We label the momenta according to
%
$$
\underline{\chi}(p) \psi(q_\psi) \leftrightarrow \chi(q_1)\chi(q_2) \; ,
$$
in the process where we keep track of the $\chi$ that appears in the same state as $\psi$, and 
%
$$
\chi(q_\chi) \psi(q_\psi) \leftrightarrow \underline{\chi}(p)\chi(q_1) \; ,
$$
%
in the process where keep track of the $\chi$ appearing in the same state as another $\chi$. Notice that the underline indicates the particle at a definite momentum (we integrate over the others). One may think that we double-count the $\chi$ states produced, but we insist that this is not the case. The reason is that there is no clear way to choose which is the one which correctly keeps track of $\chi$.~\footnote{Notice also that we don't distinguish between the $\chi$s, we just distinguish when they appear in the interaction. This is the reason we don't include both   $\underline{\chi} \chi \to \chi \psi$ and $\chi\underline{\chi}  \to \chi \psi$. }  If we only choose the first one, we get a different equation than if we only choose the second. So these two are physically distinguishable, and they both need to be included. The BE for $f_\chi$, then, reads
%
\begin{eqnarray}
\dfrac{d}{dt} f_{\chi }(\vec{p}) &= \dfrac{1}{2E_p} \lrBiggsb{
	-\dint d\Pi_{\psi} \lrb{\dfrac{1}{2} d\Phi_{1} d\Phi_{2}}   \tM{\underline{\chi} \psi}{\chi \chi } f_{\chi }(\vec{p} )
	+ \dint d\Pi_{\chi} d\Pi_{\psi} d\Phi_{1}  \tM{\chi \psi }{\underline{\chi} \chi} \lrb{1\pm f_{\chi }(\vec{p})} 
	\nonumber \\
	&- \dint d\Pi_{1} d\Phi_{\chi} d\Phi_{\psi}   \tM{\underline{\chi} \chi}{\chi \psi } f_{\chi }(\vec{p})
	+ \dint \lrb{\dfrac{1}{2} d\Pi_{1} d\Pi_{2}} d\Phi_{\psi}  \tM{\chi \chi }{\underline{\chi} \psi} \lrb{1\pm f_{\chi }(\vec{p})}
} \;,
%
\label{eq:BE_22-chi}
\end{eqnarray}  
%
where we have denoted the $\chi$ we keep track in each term by an underscore in the matrix element. It should be clear at this point that all four terms contribute in a different way to $f_\chi$'s evolution, and they should be included. However, it would be very useful if one could actually determine if this is somehow correct, or at the very least not false.
%
To do this, we note that this system is by definition a closed system, and so its energy should be conserved. Furthermore, the energy conservation should not be  an emergent statistical effect, but it should be enforced by the 4-momentum conservation in each process separately. To check that this this is true, we integrate both~\eqs{eq:BE_22-psi,eq:BE_22-chi} over 
$\dint \dfrac{d^3 p}{ (2\pi)^3} E_p$, in order to get the evolution of the respective energy densities. Assuming that the system evolves in a constant volume (the non-constant volume would only affect the \rhs of the BE, so this is just a simplification on notation), the above argument takes the form
%
$$
\dfrac{d\rho_\chi}{dt} + \dfrac{d\rho_\psi}{dt} = 0 \;, 
$$
%
with
%
\begin{equation}
\dfrac{d\rho_\psi}{dt} = 
\dfrac{1}{2} \dint d\Pi_{1}d\Pi_{2} d\Phi_{\chi} d\Phi_{\psi} \tM{\chi\chi}{\chi \psi} \ E_{\psi}  
-\dfrac{1}{2} \dint  d\Pi_{\chi} d\Pi_{\psi} d\Phi_{1}d\Phi_{2}\tM{\chi \psi}{\chi\chi} \ E_{\psi}  
\label{eq:rho_phi_22}
\end{equation}
%
and 
%
\begin{eqnarray}
\dfrac{d\rho_\chi}{dt} &= \dfrac{1}{2} \lrBiggsb{
	-\dint d\Pi_{\psi} d\Pi_{\chi}  d\Phi_{1} d\Phi_{2}   \tM{\underline{\chi} \psi}{\chi \chi } \ E_{\chi}
	+ 2\dint d\Pi_{\chi} d\Pi_{\psi} d\Phi_{1}d\Phi_{2}  \tM{\chi \psi }{\underline{\chi} \chi} \ E_{q_2} 
	\nonumber \\
	&- 2\dint d\Pi_{1}d\Pi_{2} d\Phi_{\chi} d\Phi_{\psi}   \tM{\underline{\chi} \chi}{\chi \psi } \ E_{q_2}
	+ \dint  d\Pi_{1} d\Pi_{2} d\Phi_{\psi} d\Phi_{\chi}  \tM{\chi \chi }{\underline{\chi} \psi} \ E_{\chi}
} \;.
\label{eq:rho_chi_22}
\end{eqnarray}
%
The above equation, then give us 
%
\begin{eqnarray}
\dfrac{d\rho_\chi}{dt} + \dfrac{d\rho_\psi}{dt} &=   \dfrac{1}{2} \lrBiggsb{
\dint d\Pi_{1}d\Pi_{2} d\Phi_{\chi} d\Phi_{\psi} \tM{\chi\chi}{\chi \psi} \lrb{ -2 E_{q_2} + E_\chi + E_\psi } \\
&+
\dint d\Pi_{\chi} d\Pi_{\psi} d\Phi_{1}d\Phi_{2} \tM{\chi \psi }{\chi \chi} \lrb{2 E_{q_2} -E_\chi -E_\psi  } 
} = 0 \;,
\end{eqnarray}
%
where we note that each process vanishes due to the conservation of 4-momentum. Also notice, that if we didn't include all four terms in~\eqs{eq:BE_22-psi}, this cancellation would be impossible unless someone put some extra factors by hand after integrating the original BE.


\subsection*{Working with more complicated processes}
%
Although, at this point, it should be clear what is the correct way to treat a more general BE, it would be helpful to see how this is done. To do this, consider the same system as before, but with the only allowed processes $\chi \psi \leftrightarrow \chi \chi \chi$. Following the same reasoning as before, the BE for $f_\psi$ is 
%
\begin{eqnarray}
\dfrac{d}{dt} f_{\psi }(\vec{p}) = \dfrac{1}{2E_p} \lrBiggb {&
	\dint  \lrb{\dfrac{1}{3!} 
		d\Pi_{1} d\Pi_{2}d\Pi_{3} } d\Phi_{\chi}  \tM{\chi \chi\chi }{\chi \psi} \lrb{1\pm f_{\psi }(\vec{p})} \nonumber \\
	-&\dint d\Pi_{\chi} \lrb{\dfrac{1}{3!} d\Phi_{1} d\Phi_{2}d\Phi_{3}}   \tM{\chi \psi}{\chi \chi\chi } f_{\psi }(\vec{p} )
}\;.
\label{eq:BE_23-psi}
\end{eqnarray}
%
This is not surprising, as the only thing that changes is that more $\chi$s appear. This results to an extra phase-space volume element and a different symmetry factor to take this element into account. 

The situation is similar  also for the BE for $f_\chi$, which reads
%
\begin{eqnarray}\hspace{-0.95cm}
\dfrac{d}{dt} f_{\chi }(\vec{p}) &= \dfrac{1}{2E_p} \lrBiggsb{
	-\dint d\Pi_{\psi} \lrb{\dfrac{1}{3!} d\Phi_{1} d\Phi_{2}d\Phi_{3}}   \tM{\underline{\chi} \psi}{\chi\chi \chi } f_{\chi }(\vec{p} )
	+ \dint d\Pi_{\chi} d\Pi_{\psi} \lrb{\dfrac{1}{2}d\Phi_{1}d\Phi_{2}}  \tM{\chi \psi }{\underline{\chi} \chi\chi} \lrb{1\pm f_{\chi }(\vec{p})} 
	\nonumber \\
	&- \dint \lrb{\dfrac{1}{2}d\Pi_{1}d\Pi_{2}}  d\Phi_{\chi} d\Phi_{\psi}   \tM{\underline{\chi} \chi\chi}{\chi \psi } f_{\chi }(\vec{p})
	+ \dint \lrb{\dfrac{1}{3!} d\Pi_{1} d\Pi_{2}d\Pi_{3}}  d\Phi_{\psi}  \tM{\chi\chi \chi }{\underline{\chi} \psi} \lrb{1\pm f_{\chi }(\vec{p})}
} \;,
%
\label{eq:BE_23-chi}
\end{eqnarray}  
%
where we see again that we only need to include one additional phase-space element. it should be also be obvious that the energy is conserved exactly in the same way as in the previous example. Although this may be a trivial observation, we note that since the energy has to be conserved at each process separately, we should be able to determine if this is true without integrating. 
%
That is, for the process $\chi \chi \chi \to \chi \psi$, the energy change in $\psi$ is $E_{\psi}$, and this term comes with a symmetry factor of $\dfrac{1}{3!}$. The energy change in $\chi$ from such processes is given by third and forth terms in~\eqs{eq:BE_23-chi}. The energy change coming from the third term is $-E_{3}$ with a symmetry factor $\dfrac{1}{2}$, while the forth term gives $E_{\chi}$ with a symmetry factor $\dfrac{1}{3!}$. Overall, the net energy change in the entire system by this process is 
$$
\dfrac{1}{3!} E_\psi + \dfrac{1}{3!} E_\chi - \dfrac{1}{2} E_3 =  \dfrac{1}{3!} \lrb{E_\psi+E_\chi  - 3 E_3 } = 0 \;.
$$      
%
Similarly, the net energy change for $\chi \psi \to \chi \chi \chi$  is 
$$
-\dfrac{1}{3!} E_\psi -\dfrac{1}{3!} E_\chi + \dfrac{1}{2} E_3 = \dfrac{1}{3!} \lrb{3 E_3 - E_\psi - E_\chi  } = 0 \;.
$$

As a closing remark, we note that this procedure should hold for more complected processes and in complete models. For example, consider a case where the interaction allowed is $ \psi \chi \leftrightarrow \psi  \chi \chi \chi$. In this case, both the BE for $f_\psi$ and the BE for $f_\chi$ should have four terms each one tracking a   particle that is either created or destroyed. Schematically the BE for $f_\psi$ should look-like
%
$$\hspace{-0.9cm}
2E_p \ \dfrac{d}{dt} f_{\psi }(\vec{p}) = 
\lrBiggcb{\psi \chi \to \underline{\psi}  \lrb{\dfrac{1}{3!}\chi \chi \chi}} 
-\lrBiggcb{\underline{\psi} \chi \to \psi  \lrb{\dfrac{1}{3!}\chi \chi \chi}} 
+\lrBiggcb{ \psi \lrb{\dfrac{1}{3!}\chi \chi \chi} \to \underline{\psi}  \chi}
-\lrBiggcb{ \underline{\psi}  \lrb{\dfrac{1}{3!}\chi \chi \chi} \to \psi \chi} \;,
$$  
%
where we include the relevant symmetry factors that we need to include in the collision integral.
Similarly, the BE for $f_\chi$ of the form
%
$$\hspace{-0.9cm}
2E_p \ \dfrac{d}{dt} f_{\chi }(\vec{p}) = 
\lrBiggcb{\psi \chi \to \psi  \lrb{\dfrac{1}{2} \chi \chi} \underline{\chi} } 
-\lrBiggcb{\psi \underline{\chi}  \to \psi  \lrb{\dfrac{1}{3!}\chi \chi \chi}} 
+\lrBiggcb{ \psi  \lrb{\dfrac{1}{3!}\chi \chi \chi}  \to \psi \underline{\chi}}
-\lrBiggcb{ \psi \lrb{\dfrac{1}{2}\chi \chi}\underline{\chi} \to \psi  \chi}
 \;,
$$  
%

Observing that the first two terms in both BEs correspond to the process $\psi \chi \to \psi \chi \chi \chi $, we can determine the net energy change (including the symmetry factors) in the system (we label $E_{\psi_{1,2}}$ the energies for the  $\underline{\psi}$ that appear in initial and final states respectively) as
%
$$
\dfrac{1}{3!} E_{\psi_2} -  \dfrac{1}{3!} E_{\psi_1} + \dfrac{1}{2} E_3 - \dfrac{1}{3!} E_{\chi} =
\dfrac{1}{3!} \lrb{ E_{\psi_2} + 3 E_3 -  E_{\psi_1} -  E_{\chi}} =0 \;,
$$
%
while it is trivial to show that the energy is conserved also in the other process.



%\newpage
%%%%%%%%%%%%%%%%%%%%%%%%%%%%%%%%
%\bibliography{refs}{}
%\bibliographystyle{JHEP}                        

\end{document}
